% Options for packages loaded elsewhere
\PassOptionsToPackage{unicode}{hyperref}
\PassOptionsToPackage{hyphens}{url}
\PassOptionsToPackage{dvipsnames,svgnames,x11names}{xcolor}
%
\documentclass[
]{report}

\usepackage{amsmath,amssymb}
\usepackage{iftex}
\ifPDFTeX
  \usepackage[T1]{fontenc}
  \usepackage[utf8]{inputenc}
  \usepackage{textcomp} % provide euro and other symbols
\else % if luatex or xetex
  \usepackage{unicode-math}
  \defaultfontfeatures{Scale=MatchLowercase}
  \defaultfontfeatures[\rmfamily]{Ligatures=TeX,Scale=1}
\fi
\usepackage{lmodern}
\ifPDFTeX\else  
    % xetex/luatex font selection
\fi
% Use upquote if available, for straight quotes in verbatim environments
\IfFileExists{upquote.sty}{\usepackage{upquote}}{}
\IfFileExists{microtype.sty}{% use microtype if available
  \usepackage[]{microtype}
  \UseMicrotypeSet[protrusion]{basicmath} % disable protrusion for tt fonts
}{}
\makeatletter
\@ifundefined{KOMAClassName}{% if non-KOMA class
  \IfFileExists{parskip.sty}{%
    \usepackage{parskip}
  }{% else
    \setlength{\parindent}{0pt}
    \setlength{\parskip}{6pt plus 2pt minus 1pt}}
}{% if KOMA class
  \KOMAoptions{parskip=half}}
\makeatother
\usepackage{xcolor}
\setlength{\emergencystretch}{3em} % prevent overfull lines
\setcounter{secnumdepth}{-\maxdimen} % remove section numbering
% Make \paragraph and \subparagraph free-standing
\ifx\paragraph\undefined\else
  \let\oldparagraph\paragraph
  \renewcommand{\paragraph}[1]{\oldparagraph{#1}\mbox{}}
\fi
\ifx\subparagraph\undefined\else
  \let\oldsubparagraph\subparagraph
  \renewcommand{\subparagraph}[1]{\oldsubparagraph{#1}\mbox{}}
\fi

\usepackage{color}
\usepackage{fancyvrb}
\newcommand{\VerbBar}{|}
\newcommand{\VERB}{\Verb[commandchars=\\\{\}]}
\DefineVerbatimEnvironment{Highlighting}{Verbatim}{commandchars=\\\{\}}
% Add ',fontsize=\small' for more characters per line
\usepackage{framed}
\definecolor{shadecolor}{RGB}{241,243,245}
\newenvironment{Shaded}{\begin{snugshade}}{\end{snugshade}}
\newcommand{\AlertTok}[1]{\textcolor[rgb]{0.68,0.00,0.00}{#1}}
\newcommand{\AnnotationTok}[1]{\textcolor[rgb]{0.37,0.37,0.37}{#1}}
\newcommand{\AttributeTok}[1]{\textcolor[rgb]{0.40,0.45,0.13}{#1}}
\newcommand{\BaseNTok}[1]{\textcolor[rgb]{0.68,0.00,0.00}{#1}}
\newcommand{\BuiltInTok}[1]{\textcolor[rgb]{0.00,0.23,0.31}{#1}}
\newcommand{\CharTok}[1]{\textcolor[rgb]{0.13,0.47,0.30}{#1}}
\newcommand{\CommentTok}[1]{\textcolor[rgb]{0.37,0.37,0.37}{#1}}
\newcommand{\CommentVarTok}[1]{\textcolor[rgb]{0.37,0.37,0.37}{\textit{#1}}}
\newcommand{\ConstantTok}[1]{\textcolor[rgb]{0.56,0.35,0.01}{#1}}
\newcommand{\ControlFlowTok}[1]{\textcolor[rgb]{0.00,0.23,0.31}{#1}}
\newcommand{\DataTypeTok}[1]{\textcolor[rgb]{0.68,0.00,0.00}{#1}}
\newcommand{\DecValTok}[1]{\textcolor[rgb]{0.68,0.00,0.00}{#1}}
\newcommand{\DocumentationTok}[1]{\textcolor[rgb]{0.37,0.37,0.37}{\textit{#1}}}
\newcommand{\ErrorTok}[1]{\textcolor[rgb]{0.68,0.00,0.00}{#1}}
\newcommand{\ExtensionTok}[1]{\textcolor[rgb]{0.00,0.23,0.31}{#1}}
\newcommand{\FloatTok}[1]{\textcolor[rgb]{0.68,0.00,0.00}{#1}}
\newcommand{\FunctionTok}[1]{\textcolor[rgb]{0.28,0.35,0.67}{#1}}
\newcommand{\ImportTok}[1]{\textcolor[rgb]{0.00,0.46,0.62}{#1}}
\newcommand{\InformationTok}[1]{\textcolor[rgb]{0.37,0.37,0.37}{#1}}
\newcommand{\KeywordTok}[1]{\textcolor[rgb]{0.00,0.23,0.31}{#1}}
\newcommand{\NormalTok}[1]{\textcolor[rgb]{0.00,0.23,0.31}{#1}}
\newcommand{\OperatorTok}[1]{\textcolor[rgb]{0.37,0.37,0.37}{#1}}
\newcommand{\OtherTok}[1]{\textcolor[rgb]{0.00,0.23,0.31}{#1}}
\newcommand{\PreprocessorTok}[1]{\textcolor[rgb]{0.68,0.00,0.00}{#1}}
\newcommand{\RegionMarkerTok}[1]{\textcolor[rgb]{0.00,0.23,0.31}{#1}}
\newcommand{\SpecialCharTok}[1]{\textcolor[rgb]{0.37,0.37,0.37}{#1}}
\newcommand{\SpecialStringTok}[1]{\textcolor[rgb]{0.13,0.47,0.30}{#1}}
\newcommand{\StringTok}[1]{\textcolor[rgb]{0.13,0.47,0.30}{#1}}
\newcommand{\VariableTok}[1]{\textcolor[rgb]{0.07,0.07,0.07}{#1}}
\newcommand{\VerbatimStringTok}[1]{\textcolor[rgb]{0.13,0.47,0.30}{#1}}
\newcommand{\WarningTok}[1]{\textcolor[rgb]{0.37,0.37,0.37}{\textit{#1}}}

\providecommand{\tightlist}{%
  \setlength{\itemsep}{0pt}\setlength{\parskip}{0pt}}\usepackage{longtable,booktabs,array}
\usepackage{calc} % for calculating minipage widths
% Correct order of tables after \paragraph or \subparagraph
\usepackage{etoolbox}
\makeatletter
\patchcmd\longtable{\par}{\if@noskipsec\mbox{}\fi\par}{}{}
\makeatother
% Allow footnotes in longtable head/foot
\IfFileExists{footnotehyper.sty}{\usepackage{footnotehyper}}{\usepackage{footnote}}
\makesavenoteenv{longtable}
\usepackage{graphicx}
\makeatletter
\def\maxwidth{\ifdim\Gin@nat@width>\linewidth\linewidth\else\Gin@nat@width\fi}
\def\maxheight{\ifdim\Gin@nat@height>\textheight\textheight\else\Gin@nat@height\fi}
\makeatother
% Scale images if necessary, so that they will not overflow the page
% margins by default, and it is still possible to overwrite the defaults
% using explicit options in \includegraphics[width, height, ...]{}
\setkeys{Gin}{width=\maxwidth,height=\maxheight,keepaspectratio}
% Set default figure placement to htbp
\makeatletter
\def\fps@figure{htbp}
\makeatother

\usepackage{fontspec}
\usepackage{multirow}
\usepackage{multicol}
\usepackage{colortbl}
\usepackage{hhline}
\newlength\Oldarrayrulewidth
\newlength\Oldtabcolsep
\usepackage{longtable}
\usepackage{array}
\usepackage{hyperref}
\usepackage{float}
\usepackage{wrapfig}
\makeatletter
\@ifpackageloaded{caption}{}{\usepackage{caption}}
\AtBeginDocument{%
\ifdefined\contentsname
  \renewcommand*\contentsname{Table des matières}
\else
  \newcommand\contentsname{Table des matières}
\fi
\ifdefined\listfigurename
  \renewcommand*\listfigurename{Liste des Figures}
\else
  \newcommand\listfigurename{Liste des Figures}
\fi
\ifdefined\listtablename
  \renewcommand*\listtablename{Liste des Tables}
\else
  \newcommand\listtablename{Liste des Tables}
\fi
\ifdefined\figurename
  \renewcommand*\figurename{Figure}
\else
  \newcommand\figurename{Figure}
\fi
\ifdefined\tablename
  \renewcommand*\tablename{Table}
\else
  \newcommand\tablename{Table}
\fi
}
\@ifpackageloaded{float}{}{\usepackage{float}}
\floatstyle{ruled}
\@ifundefined{c@chapter}{\newfloat{codelisting}{h}{lop}}{\newfloat{codelisting}{h}{lop}[chapter]}
\floatname{codelisting}{Listing}
\newcommand*\listoflistings{\listof{codelisting}{Liste des Listings}}
\makeatother
\makeatletter
\makeatother
\makeatletter
\@ifpackageloaded{caption}{}{\usepackage{caption}}
\@ifpackageloaded{subcaption}{}{\usepackage{subcaption}}
\makeatother
\ifLuaTeX
\usepackage[bidi=basic]{babel}
\else
\usepackage[bidi=default]{babel}
\fi
\babelprovide[main,import]{french}
% get rid of language-specific shorthands (see #6817):
\let\LanguageShortHands\languageshorthands
\def\languageshorthands#1{}
\ifLuaTeX
  \usepackage{selnolig}  % disable illegal ligatures
\fi
\usepackage{bookmark}

\IfFileExists{xurl.sty}{\usepackage{xurl}}{} % add URL line breaks if available
\urlstyle{same} % disable monospaced font for URLs
\hypersetup{
  pdftitle={Relations entre morphologie et stratégies de nage chez les poissons des Grands Lacs},
  pdfauthor={Amal Abarou, Ide Tchuileng, Christelle Baseka},
  pdflang={fr},
  colorlinks=true,
  linkcolor={blue},
  filecolor={Maroon},
  citecolor={Blue},
  urlcolor={Blue},
  pdfcreator={LaTeX via pandoc}}

\title{Relations entre morphologie et stratégies de nage chez les
poissons des Grands Lacs}
\author{Amal Abarou, Ide Tchuileng, Christelle Baseka}
\date{}

\begin{document}
\maketitle

\renewcommand*\contentsname{Table des matières}
{
\hypersetup{linkcolor=}
\setcounter{tocdepth}{2}
\tableofcontents
}
\begin{Shaded}
\begin{Highlighting}[]
\CommentTok{\# Configuration de l\textquotesingle{}environnement}
\NormalTok{SciViews}\SpecialCharTok{::}\FunctionTok{R}\NormalTok{(}\StringTok{"model"}\NormalTok{,}\StringTok{"infer"}\NormalTok{, }\AttributeTok{lang =} \StringTok{"fr"}\NormalTok{)}
\CommentTok{\# Importation des données}
\NormalTok{Fish\_data }\OtherTok{\textless{}{-}} \FunctionTok{read.csv}\NormalTok{(}\StringTok{"data/Fish\_data.csv"}\NormalTok{, }\AttributeTok{header =} \ConstantTok{TRUE}\NormalTok{, }\AttributeTok{sep =} \StringTok{","}\NormalTok{)}
\CommentTok{\#nouvelle variable binaire}
\NormalTok{Fish\_data}\SpecialCharTok{$}\NormalTok{Comportement\_banc\_bin }\OtherTok{\textless{}{-}} \FunctionTok{ifelse}\NormalTok{(Fish\_data}\SpecialCharTok{$}\NormalTok{Comportement\_banc }\SpecialCharTok{==} \StringTok{"schooling"}\NormalTok{, }\DecValTok{1}\NormalTok{, }\DecValTok{0}\NormalTok{)}
\end{Highlighting}
\end{Shaded}

\section{Introduction}\label{introduction}

La morphologie des poissons constitue un déterminant majeur de leur
écologie de déplacement, influençant la performance de nage,
l'efficacité énergétique, les interactions sociales ainsi que la
capacité à franchir des obstacles physiques tels que des rapides ou des
passes à poissons. Les traits morphologiques tels que la forme du corps,
la profondeur relative, le rapport d'aspect de la nageoire caudale ou
encore la taille des nageoires pectorales sont directement associés aux
stratégies de nage, qu'elles soient axées sur la vitesse, l'endurance,
la manoeuvrabilité ou la stabilité dans la colonne d'eau. De même,
certains traits comportementaux, tels que la tendance à former des
bancs, constituent des réponses écologiques façonnées par le régime
hydrodynamique, la prédation ou les modes d'alimentation.

Dans ce contexte, la base de données FishPass Sortable Attribute
Database (Benoit et al., 2023) constitue une ressource pertinente,
regroupant des informations morphologiques, physiologiques,
comportementales et phénologiques pour un large ensemble d'espèces des
Grands Lacs. Elle permet notamment d'analyser les facteurs influençant
la capacité des poissons à se déplacer et à franchir des aménagements
anthropiques.

Pour ce projet, nous avons combiné deux ensembles de données issus de
cette base : FishPass Morphology et FishPass Behaviour. L'objectif est
d'étudier les relations entre traits morphologiques et comportements de
nage, en particulier la tendance à former des bancs et la position
verticale dans la colonne d'eau. Les variables retenues incluent des
indicateurs de forme corporelle (profondeur du corps, rapport d'aspect,
étroitesse du pédoncule caudal) ainsi que des traits fonctionnels liés à
la nage (position et taille des nageoires, position et taille de
l'oeil).

Cette étude vise à déterminer si certains profils morphologiques sont
associés à des stratégies de déplacement collectives ou individuelles,
et dans quelle mesure la morphologie peut prédire le comportement. Une
meilleure compréhension de ces relations éclaire les mécanismes
sous-jacents à l'adaptation écologique et peut guider l'aménagement de
dispositifs favorisant la migration des poissons dans les systèmes
fluviaux aménagés.

\section{But}\label{but}

L'objectif de cette étude est d'examiner comment les traits
morphologiques des poissons sont associés à la forme du corps, à la
position verticale dans la colonne d'eau et au comportement de banc, et
de déterminer dans quelle mesure la morphologie peut prédire ces
stratégies de nage.

\section{Matériel et méthodes}\label{matuxe9riel-et-muxe9thodes}

\textbf{Matériel :}

Le jeu de données utilisé provient de la base FishPass Sortable
Attribute Database, rassemblée par Benoit \emph{et al.} (2023). Cette
base regroupe des informations morphologiques, physiologiques,
comportementales et phénologiques pour les poissons des Grands Lacs.\\
Dans ce projet, nous avons plus spécifiquement exploité les fichiers
FishPass\_Morphology\_Database.csv et FishPass\_Behaviour\_Database.csv.

La base combinée couvre 220 espèces, réparties initialement sur 21
ordres taxonomiques et comprenant 21 variables biologiques. Afin
d'étudier le lien entre morphologie et stratégies de nage, nous nous
sommes concentrés sur un ensemble restreint mais pertinent de variables
morphologiques et comportementales.

\subsection{Variables quantitatives retenues (8)
:}\label{variables-quantitatives-retenues-8}

\begin{itemize}
\item
  Longueur maximale (cm) : taille maximale observée de l'espèce.
\item
  Profondeur du corps (\% TL) : rapport entre la profondeur maximale et
  la longueur totale du corps.
\item
  Rapport d'aspect : indicateur de la forme et de la performance de la
  nageoire caudale.
\item
  Étroitesse du pédoncule caudal : finesse de la jonction entre le corps
  et la nageoire caudale.
\item
  Position verticale de la nageoire pectorale
\item
  Taille de la nageoire pectorale
\item
  Position verticale de l'oeil
\item
  Taille de l'oeil (\% HL) : diamètre de l'œil rapporté à la longueur de
  la tête.
\end{itemize}

\subsection{Variables qualitatives retenues (4)
:}\label{variables-qualitatives-retenues-4}

\begin{itemize}
\item
  Ordre : classification taxonomique de l'espèce.
\item
  Forme du corps : fusiform, elongated et short/deep.
\item
  Comportement de banc : schooling et non-schooling.
\item
  Position verticale dans la colonne d'eau : benthopelagic, dermesal et
  pelagic
\end{itemize}

\subsubsection{Méthodes}\label{muxe9thodes}

\begin{itemize}
\item
  \textbf{Fusion des données} : Les bases Morphology et Behaviour ont
  été combinées par espèce en utilisant les clés: \texttt{Order},
  \texttt{Family}, \texttt{Genus}, \texttt{Scientific.Name} et
  \texttt{Common.Name}.
\item
  \textbf{Nettoyage} : renommage des colonnes en français, conversion
  des variables numériques, suppression des lignes avec valeurs
  manquantes, conversion en \texttt{factor} pour les variables
  qualitatives et ajout de labels et unités.
\item
  \textbf{Filtrage des ordres} : afin d'assurer une représentativité
  statistique suffisante, seuls les ordres présentant un effectif ≥ 10
  individus ont été conservés. Les ordres retenus pour les analyses sont
  \emph{Cypriniformes, Perciformes et Salmoniformes.} Les autres ordres
  ont été exclus pour éviter les biais liés aux faibles tailles
  d'échantillon.

  \textbf{Traitement des données :}

  \textbf{Logiciel :} Les analyses statistiques sont réalisées dans R
  version 4.4.1 (2024-06-14) et plus précisément avec
  \texttt{\{tidyverse\}} version 2.0.0, \texttt{\{tabularise\}} version
  0.7.0 et \texttt{\{labelled\}} version 2.14.0 sous Ubuntu 22.04.5 LTS.
  Le seuil α est fixé à 5 \%.
\end{itemize}

\section{Résultats}\label{ruxe9sultats}

\subsection{\texorpdfstring{\textbf{Relation entre la longueur maximale
et la taille relative de l'œil (modèle linéaire
polynomial):}}{Relation entre la longueur maximale et la taille relative de l'œil (modèle linéaire polynomial):}}\label{relation-entre-la-longueur-maximale-et-la-taille-relative-de-lux153il-moduxe8le-linuxe9aire-polynomial}

Un modèle polynomial d'ordre deux a été ajusté pour décrire la relation
entre la taille maximale du poisson (Longueur\_max) et la taille
relative de l'oeil (Taille\_oeil).

\begin{Shaded}
\begin{Highlighting}[]
\NormalTok{Poly1\_lm }\OtherTok{\textless{}{-}} \FunctionTok{lm}\NormalTok{(}\AttributeTok{data =}\NormalTok{ Fish\_data, Longueur\_max }\SpecialCharTok{\textasciitilde{}}\NormalTok{ Taille\_oeil }\SpecialCharTok{+} \FunctionTok{I}\NormalTok{(Taille\_oeil}\SpecialCharTok{\^{}}\DecValTok{2}\NormalTok{))}
\FunctionTok{summary\_}\NormalTok{(Poly1\_lm) }\SpecialCharTok{|\textgreater{}} \FunctionTok{tabularise}\NormalTok{()}
\end{Highlighting}
\end{Shaded}

\begin{verbatim}
Warning in set2(resolve(...)): The object is read-only and cannot be modified.
If you have to modify it for a legitimate reason, call the method $lock(FALSE)
on the object before $set(). Using $lock(FALSE) to modify the object will be
enforced in future versions of knitr and this warning will become an error.
\end{verbatim}

\global\setlength{\Oldarrayrulewidth}{\arrayrulewidth}

\global\setlength{\Oldtabcolsep}{\tabcolsep}

\setlength{\tabcolsep}{2pt}

\renewcommand*{\arraystretch}{1.5}



\providecommand{\ascline}[3]{\noalign{\global\arrayrulewidth #1}\arrayrulecolor[HTML]{#2}\cline{#3}}

\begin{longtable*}[c]{|p{0.95in}|p{1.40in}|p{1.04in}|p{1.09in}|p{1.14in}|p{0.40in}}



\ascline{1.5pt}{666666}{1-6}

\multicolumn{6}{>{\centering}m{\dimexpr 6.01in+10\tabcolsep}}{\textcolor[HTML]{000000}{\fontsize{12}{12}\selectfont{\global\setmainfont{Arial}{Modèle\ linéaire}}}} \\





\multicolumn{6}{>{\centering}m{\dimexpr 6.01in+10\tabcolsep}}{\textcolor[HTML]{000000}{\fontsize{12}{12}\selectfont{\global\setmainfont{Arial}{$\operatorname{Longueur_max} = \alpha + \beta_{1}(\operatorname{Taille_oeil}) + \beta_{2}(\operatorname{Taille_oeil^{2}}) + \epsilon$}}}} \\

\ascline{1.5pt}{666666}{1-6}



\multicolumn{1}{>{\raggedright}m{\dimexpr 0.95in+0\tabcolsep}}{\textcolor[HTML]{000000}{\fontsize{12}{12}\selectfont{\global\setmainfont{Arial}{}}}\textcolor[HTML]{000000}{\fontsize{12}{12}\selectfont{\global\setmainfont{Arial}{Terme}}}} & \multicolumn{1}{>{\raggedleft}m{\dimexpr 1.4in+0\tabcolsep}}{\textcolor[HTML]{000000}{\fontsize{12}{12}\selectfont{\global\setmainfont{Arial}{}}}\textcolor[HTML]{000000}{\fontsize{12}{12}\selectfont{\global\setmainfont{Arial}{Valeur\ estimée}}}} & \multicolumn{1}{>{\raggedleft}m{\dimexpr 1.04in+0\tabcolsep}}{\textcolor[HTML]{000000}{\fontsize{12}{12}\selectfont{\global\setmainfont{Arial}{}}}\textcolor[HTML]{000000}{\fontsize{12}{12}\selectfont{\global\setmainfont{Arial}{Ecart\ type}}}} & \multicolumn{1}{>{\raggedleft}m{\dimexpr 1.09in+0\tabcolsep}}{\textcolor[HTML]{000000}{\fontsize{12}{12}\selectfont{\global\setmainfont{Arial}{}}}\textcolor[HTML]{000000}{\fontsize{12}{12}\selectfont{\global\setmainfont{Arial}{Valeur\ de\ }}}\textcolor[HTML]{000000}{\fontsize{12}{12}\selectfont{\global\setmainfont{Arial}{\textit{t}}}}\textcolor[HTML]{000000}{\fontsize{12}{12}\selectfont{\global\setmainfont{Arial}{}}}} & \multicolumn{1}{>{\raggedleft}m{\dimexpr 1.14in+0\tabcolsep}}{\textcolor[HTML]{000000}{\fontsize{12}{12}\selectfont{\global\setmainfont{Arial}{}}}\textcolor[HTML]{000000}{\fontsize{12}{12}\selectfont{\global\setmainfont{Arial}{Valeur\ de\ }}}\textcolor[HTML]{000000}{\fontsize{12}{12}\selectfont{\global\setmainfont{Arial}{\textit{p}}}}\textcolor[HTML]{000000}{\fontsize{12}{12}\selectfont{\global\setmainfont{Arial}{}}}} & \multicolumn{1}{>{\raggedright}m{\dimexpr 0.4in+0\tabcolsep}}{\textcolor[HTML]{000000}{\fontsize{12}{12}\selectfont{\global\setmainfont{Arial}{}}}} \\

\ascline{1.5pt}{666666}{1-6}\endfirsthead 

\ascline{1.5pt}{666666}{1-6}

\multicolumn{6}{>{\centering}m{\dimexpr 6.01in+10\tabcolsep}}{\textcolor[HTML]{000000}{\fontsize{12}{12}\selectfont{\global\setmainfont{Arial}{Modèle\ linéaire}}}} \\





\multicolumn{6}{>{\centering}m{\dimexpr 6.01in+10\tabcolsep}}{\textcolor[HTML]{000000}{\fontsize{12}{12}\selectfont{\global\setmainfont{Arial}{$\operatorname{Longueur_max} = \alpha + \beta_{1}(\operatorname{Taille_oeil}) + \beta_{2}(\operatorname{Taille_oeil^{2}}) + \epsilon$}}}} \\

\ascline{1.5pt}{666666}{1-6}



\multicolumn{1}{>{\raggedright}m{\dimexpr 0.95in+0\tabcolsep}}{\textcolor[HTML]{000000}{\fontsize{12}{12}\selectfont{\global\setmainfont{Arial}{}}}\textcolor[HTML]{000000}{\fontsize{12}{12}\selectfont{\global\setmainfont{Arial}{Terme}}}} & \multicolumn{1}{>{\raggedleft}m{\dimexpr 1.4in+0\tabcolsep}}{\textcolor[HTML]{000000}{\fontsize{12}{12}\selectfont{\global\setmainfont{Arial}{}}}\textcolor[HTML]{000000}{\fontsize{12}{12}\selectfont{\global\setmainfont{Arial}{Valeur\ estimée}}}} & \multicolumn{1}{>{\raggedleft}m{\dimexpr 1.04in+0\tabcolsep}}{\textcolor[HTML]{000000}{\fontsize{12}{12}\selectfont{\global\setmainfont{Arial}{}}}\textcolor[HTML]{000000}{\fontsize{12}{12}\selectfont{\global\setmainfont{Arial}{Ecart\ type}}}} & \multicolumn{1}{>{\raggedleft}m{\dimexpr 1.09in+0\tabcolsep}}{\textcolor[HTML]{000000}{\fontsize{12}{12}\selectfont{\global\setmainfont{Arial}{}}}\textcolor[HTML]{000000}{\fontsize{12}{12}\selectfont{\global\setmainfont{Arial}{Valeur\ de\ }}}\textcolor[HTML]{000000}{\fontsize{12}{12}\selectfont{\global\setmainfont{Arial}{\textit{t}}}}\textcolor[HTML]{000000}{\fontsize{12}{12}\selectfont{\global\setmainfont{Arial}{}}}} & \multicolumn{1}{>{\raggedleft}m{\dimexpr 1.14in+0\tabcolsep}}{\textcolor[HTML]{000000}{\fontsize{12}{12}\selectfont{\global\setmainfont{Arial}{}}}\textcolor[HTML]{000000}{\fontsize{12}{12}\selectfont{\global\setmainfont{Arial}{Valeur\ de\ }}}\textcolor[HTML]{000000}{\fontsize{12}{12}\selectfont{\global\setmainfont{Arial}{\textit{p}}}}\textcolor[HTML]{000000}{\fontsize{12}{12}\selectfont{\global\setmainfont{Arial}{}}}} & \multicolumn{1}{>{\raggedright}m{\dimexpr 0.4in+0\tabcolsep}}{\textcolor[HTML]{000000}{\fontsize{12}{12}\selectfont{\global\setmainfont{Arial}{}}}} \\

\ascline{1.5pt}{666666}{1-6}\endhead



\multicolumn{1}{>{\raggedright}m{\dimexpr 0.95in+0\tabcolsep}}{\textcolor[HTML]{000000}{\fontsize{12}{12}\selectfont{\global\setmainfont{Arial}{}}}\textcolor[HTML]{000000}{\fontsize{12}{12}\selectfont{\global\setmainfont{Arial}{}}}\textcolor[HTML]{000000}{\fontsize{12}{12}\selectfont{\global\setmainfont{Arial}{$\alpha$}}}\textcolor[HTML]{000000}{\fontsize{12}{12}\selectfont{\global\setmainfont{Arial}{}}}} & \multicolumn{1}{>{\raggedleft}m{\dimexpr 1.4in+0\tabcolsep}}{\textcolor[HTML]{000000}{\fontsize{12}{12}\selectfont{\global\setmainfont{Arial}{271.864}}}\textcolor[HTML]{000000}{\fontsize{12}{12}\selectfont{\global\setmainfont{Arial}{}}}\textcolor[HTML]{000000}{\fontsize{12}{12}\selectfont{\global\setmainfont{Arial}{\textsuperscript{}}}}} & \multicolumn{1}{>{\raggedleft}m{\dimexpr 1.04in+0\tabcolsep}}{\textcolor[HTML]{000000}{\fontsize{12}{12}\selectfont{\global\setmainfont{Arial}{31.9234}}}\textcolor[HTML]{000000}{\fontsize{12}{12}\selectfont{\global\setmainfont{Arial}{}}}\textcolor[HTML]{000000}{\fontsize{12}{12}\selectfont{\global\setmainfont{Arial}{\textsuperscript{}}}}} & \multicolumn{1}{>{\raggedleft}m{\dimexpr 1.09in+0\tabcolsep}}{\textcolor[HTML]{000000}{\fontsize{12}{12}\selectfont{\global\setmainfont{Arial}{\ 8.52}}}\textcolor[HTML]{000000}{\fontsize{12}{12}\selectfont{\global\setmainfont{Arial}{}}}\textcolor[HTML]{000000}{\fontsize{12}{12}\selectfont{\global\setmainfont{Arial}{\textsuperscript{}}}}} & \multicolumn{1}{>{\raggedleft}m{\dimexpr 1.14in+0\tabcolsep}}{\textcolor[HTML]{000000}{\fontsize{12}{12}\selectfont{\global\setmainfont{Arial}{1.86}}}\textcolor[HTML]{000000}{\fontsize{12}{12}\selectfont{\global\setmainfont{Arial}{·10}}}\textcolor[HTML]{000000}{\fontsize{12}{12}\selectfont{\global\setmainfont{Arial}{\textsuperscript{-13}}}}} & \multicolumn{1}{>{\raggedright}m{\dimexpr 0.4in+0\tabcolsep}}{\textcolor[HTML]{000000}{\fontsize{12}{12}\selectfont{\global\setmainfont{Arial}{***}}}} \\





\multicolumn{1}{>{\raggedright}m{\dimexpr 0.95in+0\tabcolsep}}{\textcolor[HTML]{000000}{\fontsize{12}{12}\selectfont{\global\setmainfont{Arial}{}}}\textcolor[HTML]{000000}{\fontsize{12}{12}\selectfont{\global\setmainfont{Arial}{}}}\textcolor[HTML]{000000}{\fontsize{12}{12}\selectfont{\global\setmainfont{Arial}{$\beta_{1}$}}}\textcolor[HTML]{000000}{\fontsize{12}{12}\selectfont{\global\setmainfont{Arial}{}}}} & \multicolumn{1}{>{\raggedleft}m{\dimexpr 1.4in+0\tabcolsep}}{\textcolor[HTML]{000000}{\fontsize{12}{12}\selectfont{\global\setmainfont{Arial}{-15.126}}}\textcolor[HTML]{000000}{\fontsize{12}{12}\selectfont{\global\setmainfont{Arial}{}}}\textcolor[HTML]{000000}{\fontsize{12}{12}\selectfont{\global\setmainfont{Arial}{\textsuperscript{}}}}} & \multicolumn{1}{>{\raggedleft}m{\dimexpr 1.04in+0\tabcolsep}}{\textcolor[HTML]{000000}{\fontsize{12}{12}\selectfont{\global\setmainfont{Arial}{\ 2.7736}}}\textcolor[HTML]{000000}{\fontsize{12}{12}\selectfont{\global\setmainfont{Arial}{}}}\textcolor[HTML]{000000}{\fontsize{12}{12}\selectfont{\global\setmainfont{Arial}{\textsuperscript{}}}}} & \multicolumn{1}{>{\raggedleft}m{\dimexpr 1.09in+0\tabcolsep}}{\textcolor[HTML]{000000}{\fontsize{12}{12}\selectfont{\global\setmainfont{Arial}{-5.45}}}\textcolor[HTML]{000000}{\fontsize{12}{12}\selectfont{\global\setmainfont{Arial}{}}}\textcolor[HTML]{000000}{\fontsize{12}{12}\selectfont{\global\setmainfont{Arial}{\textsuperscript{}}}}} & \multicolumn{1}{>{\raggedleft}m{\dimexpr 1.14in+0\tabcolsep}}{\textcolor[HTML]{000000}{\fontsize{12}{12}\selectfont{\global\setmainfont{Arial}{3.64}}}\textcolor[HTML]{000000}{\fontsize{12}{12}\selectfont{\global\setmainfont{Arial}{·10}}}\textcolor[HTML]{000000}{\fontsize{12}{12}\selectfont{\global\setmainfont{Arial}{\textsuperscript{-07}}}}} & \multicolumn{1}{>{\raggedright}m{\dimexpr 0.4in+0\tabcolsep}}{\textcolor[HTML]{000000}{\fontsize{12}{12}\selectfont{\global\setmainfont{Arial}{***}}}} \\





\multicolumn{1}{>{\raggedright}m{\dimexpr 0.95in+0\tabcolsep}}{\textcolor[HTML]{000000}{\fontsize{12}{12}\selectfont{\global\setmainfont{Arial}{}}}\textcolor[HTML]{000000}{\fontsize{12}{12}\selectfont{\global\setmainfont{Arial}{}}}\textcolor[HTML]{000000}{\fontsize{12}{12}\selectfont{\global\setmainfont{Arial}{$\beta_{2}$}}}\textcolor[HTML]{000000}{\fontsize{12}{12}\selectfont{\global\setmainfont{Arial}{}}}} & \multicolumn{1}{>{\raggedleft}m{\dimexpr 1.4in+0\tabcolsep}}{\textcolor[HTML]{000000}{\fontsize{12}{12}\selectfont{\global\setmainfont{Arial}{\ \ 0.215}}}\textcolor[HTML]{000000}{\fontsize{12}{12}\selectfont{\global\setmainfont{Arial}{}}}\textcolor[HTML]{000000}{\fontsize{12}{12}\selectfont{\global\setmainfont{Arial}{\textsuperscript{}}}}} & \multicolumn{1}{>{\raggedleft}m{\dimexpr 1.04in+0\tabcolsep}}{\textcolor[HTML]{000000}{\fontsize{12}{12}\selectfont{\global\setmainfont{Arial}{\ 0.0583}}}\textcolor[HTML]{000000}{\fontsize{12}{12}\selectfont{\global\setmainfont{Arial}{}}}\textcolor[HTML]{000000}{\fontsize{12}{12}\selectfont{\global\setmainfont{Arial}{\textsuperscript{}}}}} & \multicolumn{1}{>{\raggedleft}m{\dimexpr 1.09in+0\tabcolsep}}{\textcolor[HTML]{000000}{\fontsize{12}{12}\selectfont{\global\setmainfont{Arial}{\ 3.69}}}\textcolor[HTML]{000000}{\fontsize{12}{12}\selectfont{\global\setmainfont{Arial}{}}}\textcolor[HTML]{000000}{\fontsize{12}{12}\selectfont{\global\setmainfont{Arial}{\textsuperscript{}}}}} & \multicolumn{1}{>{\raggedleft}m{\dimexpr 1.14in+0\tabcolsep}}{\textcolor[HTML]{000000}{\fontsize{12}{12}\selectfont{\global\setmainfont{Arial}{3.63}}}\textcolor[HTML]{000000}{\fontsize{12}{12}\selectfont{\global\setmainfont{Arial}{·10}}}\textcolor[HTML]{000000}{\fontsize{12}{12}\selectfont{\global\setmainfont{Arial}{\textsuperscript{-04}}}}} & \multicolumn{1}{>{\raggedright}m{\dimexpr 0.4in+0\tabcolsep}}{\textcolor[HTML]{000000}{\fontsize{12}{12}\selectfont{\global\setmainfont{Arial}{***}}}} \\

\ascline{1.5pt}{666666}{1-6}



\multicolumn{6}{>{\raggedleft}m{\dimexpr 6.01in+10\tabcolsep}}{\textcolor[HTML]{000000}{\fontsize{12}{12}\selectfont{\global\setmainfont{Arial}{0\ <=\ '***'\ <\ 0.001\ <\ '**'\ <\ 0.01\ <\ '*'\ <\ 0.05}}}} \\





\multicolumn{6}{>{\raggedright}m{\dimexpr 6.01in+10\tabcolsep}}{\textcolor[HTML]{000000}{\fontsize{12}{12}\selectfont{\global\setmainfont{Arial}{Etendue\ des\ résidus\ :\ [-95.43,\ 105.6]}}}\textcolor[HTML]{000000}{\fontsize{12}{12}\selectfont{\global\setmainfont{Arial}{\linebreak }}}\textcolor[HTML]{000000}{\fontsize{12}{12}\selectfont{\global\setmainfont{Arial}{Ecart\ type\ des\ résidus\ :\ 30.85\ pour\ 99\ degrés\ de\ liberté}}}\textcolor[HTML]{000000}{\fontsize{12}{12}\selectfont{\global\setmainfont{Arial}{\linebreak }}}\textcolor[HTML]{000000}{\fontsize{12}{12}\selectfont{\global\setmainfont{Arial}{\textit{R}}}}\textcolor[HTML]{000000}{\fontsize{12}{12}\selectfont{\global\setmainfont{Arial}{\textsuperscript{2}}}}\textcolor[HTML]{000000}{\fontsize{12}{12}\selectfont{\global\setmainfont{Arial}{\ multiple\ :\ 0.5452\ -\ }}}\textcolor[HTML]{000000}{\fontsize{12}{12}\selectfont{\global\setmainfont{Arial}{\textit{R}}}}\textcolor[HTML]{000000}{\fontsize{12}{12}\selectfont{\global\setmainfont{Arial}{\textsuperscript{2}}}}\textcolor[HTML]{000000}{\fontsize{12}{12}\selectfont{\global\setmainfont{Arial}{\ ajusté\ :\ 0.536}}}\textcolor[HTML]{000000}{\fontsize{12}{12}\selectfont{\global\setmainfont{Arial}{\linebreak }}}\textcolor[HTML]{000000}{\fontsize{12}{12}\selectfont{\global\setmainfont{Arial}{Statistique\ }}}\textcolor[HTML]{000000}{\fontsize{12}{12}\selectfont{\global\setmainfont{Arial}{\textit{F}}}}\textcolor[HTML]{000000}{\fontsize{12}{12}\selectfont{\global\setmainfont{Arial}{\ :\ 59.35\ sur\ 2\ et\ 99\ ddl\ -\ valeur\ de\ }}}\textcolor[HTML]{000000}{\fontsize{12}{12}\selectfont{\global\setmainfont{Arial}{\textit{p}}}}\textcolor[HTML]{000000}{\fontsize{12}{12}\selectfont{\global\setmainfont{Arial}{\ :\ <\ 2.22e-16}}}} \\





\end{longtable*}



\arrayrulecolor[HTML]{000000}

\global\setlength{\arrayrulewidth}{\Oldarrayrulewidth}

\global\setlength{\tabcolsep}{\Oldtabcolsep}

\renewcommand*{\arraystretch}{1}

Le modèle s'ajuste de manière satisfaisante aux données morphologiques.
Les coefficients des deux termes sont significativement différents de
zéro au seuil de 5 \%, indiquant une relation non linéaire entre les
deux variables. Le coefficient de détermination (R² ajusté = 0,536)
montre qu'un peu plus de la moitié de la variation de la taille relative
de l'oeil est expliquée par la longueur maximale du poisson.

\begin{Shaded}
\begin{Highlighting}[]
\CommentTok{\#Le graphique du modèle (Fig. X)}
\FunctionTok{chart}\NormalTok{(Poly1\_lm)}
\end{Highlighting}
\end{Shaded}

\includegraphics{models_report_files/figure-pdf/residus modele1-1.pdf}

\begin{Shaded}
\begin{Highlighting}[]
\CommentTok{\#Analyse des résidus (Fig. Y)}
\NormalTok{chart}\SpecialCharTok{$}\FunctionTok{residuals}\NormalTok{(Poly1\_lm)}
\end{Highlighting}
\end{Shaded}

\includegraphics{models_report_files/figure-pdf/residus modele1-2.pdf}

Le graphique du modèle (Fig. X) montre que les espèces de petite taille
possèdent des yeux proportionnellement plus grands, tandis que les
espèces plus grandes présentent des valeurs plus faibles. Cette tendance
est cohérente avec les lois allométriques habituellement observées chez
les poissons.

L'analyse des résidus (Fig. Y) révèle une distribution globalement
conforme à la normalité, malgré une légère hétéroscédasticité pour les
grandes longueurs et quelques points influents, ce qui demeure
acceptable pour ce type de données biologiques.

\subsection{\texorpdfstring{\textbf{Déterminants morphologiques du
comportement de banc (GLM
binomial):}}{Déterminants morphologiques du comportement de banc (GLM binomial):}}\label{duxe9terminants-morphologiques-du-comportement-de-banc-glm-binomial}

Un modèle linéaire généralisé de type logistique a été ajusté afin
d'évaluer l'effet de deux variables morphologiques sur la probabilité
pour une espèce de présenter un comportement de banc
(\emph{Comportement\_banc\_bin}).

\begin{Shaded}
\begin{Highlighting}[]
\NormalTok{Fish\_glm1 }\OtherTok{\textless{}{-}} \FunctionTok{glm}\NormalTok{(}\AttributeTok{data =}\NormalTok{ Fish\_data, Comportement\_banc\_bin }\SpecialCharTok{\textasciitilde{}}\NormalTok{ Rapport\_aspect }\SpecialCharTok{+}\NormalTok{ Position\_oeil\_vertical, }\AttributeTok{family =}\NormalTok{ binomial)}
\FunctionTok{summary\_}\NormalTok{(Fish\_glm1) }\SpecialCharTok{|\textgreater{}} \FunctionTok{tabularise}\NormalTok{()}
\end{Highlighting}
\end{Shaded}

\begin{verbatim}
Warning in set2(resolve(...)): The object is read-only and cannot be modified.
If you have to modify it for a legitimate reason, call the method $lock(FALSE)
on the object before $set(). Using $lock(FALSE) to modify the object will be
enforced in future versions of knitr and this warning will become an error.
\end{verbatim}

\global\setlength{\Oldarrayrulewidth}{\arrayrulewidth}

\global\setlength{\Oldtabcolsep}{\tabcolsep}

\setlength{\tabcolsep}{2pt}

\renewcommand*{\arraystretch}{1.5}



\providecommand{\ascline}[3]{\noalign{\global\arrayrulewidth #1}\arrayrulecolor[HTML]{#2}\cline{#3}}

\begin{longtable*}[c]{|p{0.95in}|p{1.40in}|p{1.04in}|p{1.13in}|p{1.14in}}



\ascline{1.5pt}{666666}{1-5}

\multicolumn{5}{>{\centering}m{\dimexpr 5.65in+8\tabcolsep}}{\textcolor[HTML]{000000}{\fontsize{12}{12}\selectfont{\global\setmainfont{Arial}{Modèle\ linéaire\ généralisé}}}} \\





\multicolumn{5}{>{\centering}m{\dimexpr 5.65in+8\tabcolsep}}{\textcolor[HTML]{000000}{\fontsize{12}{12}\selectfont{\global\setmainfont{Arial}{$\log\left[ \frac { P( \operatorname{Comportement_banc_bin} = \operatorname{1} ) }{ 1 - P( \operatorname{Comportement_banc_bin} = \operatorname{1} ) } \right] = \alpha + \beta_{1}(\operatorname{Rapport_aspect}) + \beta_{2}(\operatorname{Position_oeil_vertical})$}}}} \\

\ascline{1.5pt}{666666}{1-5}



\multicolumn{1}{>{\raggedright}m{\dimexpr 0.95in+0\tabcolsep}}{\textcolor[HTML]{000000}{\fontsize{12}{12}\selectfont{\global\setmainfont{Arial}{}}}\textcolor[HTML]{000000}{\fontsize{12}{12}\selectfont{\global\setmainfont{Arial}{Terme}}}} & \multicolumn{1}{>{\raggedleft}m{\dimexpr 1.4in+0\tabcolsep}}{\textcolor[HTML]{000000}{\fontsize{12}{12}\selectfont{\global\setmainfont{Arial}{}}}\textcolor[HTML]{000000}{\fontsize{12}{12}\selectfont{\global\setmainfont{Arial}{Valeur\ estimée}}}} & \multicolumn{1}{>{\raggedleft}m{\dimexpr 1.04in+0\tabcolsep}}{\textcolor[HTML]{000000}{\fontsize{12}{12}\selectfont{\global\setmainfont{Arial}{}}}\textcolor[HTML]{000000}{\fontsize{12}{12}\selectfont{\global\setmainfont{Arial}{Ecart\ type}}}} & \multicolumn{1}{>{\raggedleft}m{\dimexpr 1.13in+0\tabcolsep}}{\textcolor[HTML]{000000}{\fontsize{12}{12}\selectfont{\global\setmainfont{Arial}{}}}\textcolor[HTML]{000000}{\fontsize{12}{12}\selectfont{\global\setmainfont{Arial}{Valeur\ de\ }}}\textcolor[HTML]{000000}{\fontsize{12}{12}\selectfont{\global\setmainfont{Arial}{\textit{z}}}}\textcolor[HTML]{000000}{\fontsize{12}{12}\selectfont{\global\setmainfont{Arial}{}}}} & \multicolumn{1}{>{\raggedleft}m{\dimexpr 1.14in+0\tabcolsep}}{\textcolor[HTML]{000000}{\fontsize{12}{12}\selectfont{\global\setmainfont{Arial}{}}}\textcolor[HTML]{000000}{\fontsize{12}{12}\selectfont{\global\setmainfont{Arial}{Valeur\ de\ }}}\textcolor[HTML]{000000}{\fontsize{12}{12}\selectfont{\global\setmainfont{Arial}{\textit{p}}}}\textcolor[HTML]{000000}{\fontsize{12}{12}\selectfont{\global\setmainfont{Arial}{}}}} \\

\ascline{1.5pt}{666666}{1-5}\endfirsthead 

\ascline{1.5pt}{666666}{1-5}

\multicolumn{5}{>{\centering}m{\dimexpr 5.65in+8\tabcolsep}}{\textcolor[HTML]{000000}{\fontsize{12}{12}\selectfont{\global\setmainfont{Arial}{Modèle\ linéaire\ généralisé}}}} \\





\multicolumn{5}{>{\centering}m{\dimexpr 5.65in+8\tabcolsep}}{\textcolor[HTML]{000000}{\fontsize{12}{12}\selectfont{\global\setmainfont{Arial}{$\log\left[ \frac { P( \operatorname{Comportement_banc_bin} = \operatorname{1} ) }{ 1 - P( \operatorname{Comportement_banc_bin} = \operatorname{1} ) } \right] = \alpha + \beta_{1}(\operatorname{Rapport_aspect}) + \beta_{2}(\operatorname{Position_oeil_vertical})$}}}} \\

\ascline{1.5pt}{666666}{1-5}



\multicolumn{1}{>{\raggedright}m{\dimexpr 0.95in+0\tabcolsep}}{\textcolor[HTML]{000000}{\fontsize{12}{12}\selectfont{\global\setmainfont{Arial}{}}}\textcolor[HTML]{000000}{\fontsize{12}{12}\selectfont{\global\setmainfont{Arial}{Terme}}}} & \multicolumn{1}{>{\raggedleft}m{\dimexpr 1.4in+0\tabcolsep}}{\textcolor[HTML]{000000}{\fontsize{12}{12}\selectfont{\global\setmainfont{Arial}{}}}\textcolor[HTML]{000000}{\fontsize{12}{12}\selectfont{\global\setmainfont{Arial}{Valeur\ estimée}}}} & \multicolumn{1}{>{\raggedleft}m{\dimexpr 1.04in+0\tabcolsep}}{\textcolor[HTML]{000000}{\fontsize{12}{12}\selectfont{\global\setmainfont{Arial}{}}}\textcolor[HTML]{000000}{\fontsize{12}{12}\selectfont{\global\setmainfont{Arial}{Ecart\ type}}}} & \multicolumn{1}{>{\raggedleft}m{\dimexpr 1.13in+0\tabcolsep}}{\textcolor[HTML]{000000}{\fontsize{12}{12}\selectfont{\global\setmainfont{Arial}{}}}\textcolor[HTML]{000000}{\fontsize{12}{12}\selectfont{\global\setmainfont{Arial}{Valeur\ de\ }}}\textcolor[HTML]{000000}{\fontsize{12}{12}\selectfont{\global\setmainfont{Arial}{\textit{z}}}}\textcolor[HTML]{000000}{\fontsize{12}{12}\selectfont{\global\setmainfont{Arial}{}}}} & \multicolumn{1}{>{\raggedleft}m{\dimexpr 1.14in+0\tabcolsep}}{\textcolor[HTML]{000000}{\fontsize{12}{12}\selectfont{\global\setmainfont{Arial}{}}}\textcolor[HTML]{000000}{\fontsize{12}{12}\selectfont{\global\setmainfont{Arial}{Valeur\ de\ }}}\textcolor[HTML]{000000}{\fontsize{12}{12}\selectfont{\global\setmainfont{Arial}{\textit{p}}}}\textcolor[HTML]{000000}{\fontsize{12}{12}\selectfont{\global\setmainfont{Arial}{}}}} \\

\ascline{1.5pt}{666666}{1-5}\endhead



\multicolumn{1}{>{\raggedright}m{\dimexpr 0.95in+0\tabcolsep}}{\textcolor[HTML]{000000}{\fontsize{12}{12}\selectfont{\global\setmainfont{Arial}{}}}\textcolor[HTML]{000000}{\fontsize{12}{12}\selectfont{\global\setmainfont{Arial}{}}}\textcolor[HTML]{000000}{\fontsize{12}{12}\selectfont{\global\setmainfont{Arial}{$\alpha$}}}\textcolor[HTML]{000000}{\fontsize{12}{12}\selectfont{\global\setmainfont{Arial}{}}}} & \multicolumn{1}{>{\raggedleft}m{\dimexpr 1.4in+0\tabcolsep}}{\textcolor[HTML]{000000}{\fontsize{12}{12}\selectfont{\global\setmainfont{Arial}{\ \ 5.990}}}\textcolor[HTML]{000000}{\fontsize{12}{12}\selectfont{\global\setmainfont{Arial}{}}}\textcolor[HTML]{000000}{\fontsize{12}{12}\selectfont{\global\setmainfont{Arial}{\textsuperscript{}}}}} & \multicolumn{1}{>{\raggedleft}m{\dimexpr 1.04in+0\tabcolsep}}{\textcolor[HTML]{000000}{\fontsize{12}{12}\selectfont{\global\setmainfont{Arial}{2.219}}}\textcolor[HTML]{000000}{\fontsize{12}{12}\selectfont{\global\setmainfont{Arial}{}}}\textcolor[HTML]{000000}{\fontsize{12}{12}\selectfont{\global\setmainfont{Arial}{\textsuperscript{}}}}} & \multicolumn{1}{>{\raggedleft}m{\dimexpr 1.13in+0\tabcolsep}}{\textcolor[HTML]{000000}{\fontsize{12}{12}\selectfont{\global\setmainfont{Arial}{\ 2.70}}}\textcolor[HTML]{000000}{\fontsize{12}{12}\selectfont{\global\setmainfont{Arial}{}}}\textcolor[HTML]{000000}{\fontsize{12}{12}\selectfont{\global\setmainfont{Arial}{\textsuperscript{}}}}} & \multicolumn{1}{>{\raggedleft}m{\dimexpr 1.14in+0\tabcolsep}}{\textcolor[HTML]{000000}{\fontsize{12}{12}\selectfont{\global\setmainfont{Arial}{0.00694}}}\textcolor[HTML]{000000}{\fontsize{12}{12}\selectfont{\global\setmainfont{Arial}{}}}\textcolor[HTML]{000000}{\fontsize{12}{12}\selectfont{\global\setmainfont{Arial}{\textsuperscript{}}}}} \\





\multicolumn{1}{>{\raggedright}m{\dimexpr 0.95in+0\tabcolsep}}{\textcolor[HTML]{000000}{\fontsize{12}{12}\selectfont{\global\setmainfont{Arial}{}}}\textcolor[HTML]{000000}{\fontsize{12}{12}\selectfont{\global\setmainfont{Arial}{}}}\textcolor[HTML]{000000}{\fontsize{12}{12}\selectfont{\global\setmainfont{Arial}{$\beta_{1}$}}}\textcolor[HTML]{000000}{\fontsize{12}{12}\selectfont{\global\setmainfont{Arial}{}}}} & \multicolumn{1}{>{\raggedleft}m{\dimexpr 1.4in+0\tabcolsep}}{\textcolor[HTML]{000000}{\fontsize{12}{12}\selectfont{\global\setmainfont{Arial}{\ \ 0.982}}}\textcolor[HTML]{000000}{\fontsize{12}{12}\selectfont{\global\setmainfont{Arial}{}}}\textcolor[HTML]{000000}{\fontsize{12}{12}\selectfont{\global\setmainfont{Arial}{\textsuperscript{}}}}} & \multicolumn{1}{>{\raggedleft}m{\dimexpr 1.04in+0\tabcolsep}}{\textcolor[HTML]{000000}{\fontsize{12}{12}\selectfont{\global\setmainfont{Arial}{0.467}}}\textcolor[HTML]{000000}{\fontsize{12}{12}\selectfont{\global\setmainfont{Arial}{}}}\textcolor[HTML]{000000}{\fontsize{12}{12}\selectfont{\global\setmainfont{Arial}{\textsuperscript{}}}}} & \multicolumn{1}{>{\raggedleft}m{\dimexpr 1.13in+0\tabcolsep}}{\textcolor[HTML]{000000}{\fontsize{12}{12}\selectfont{\global\setmainfont{Arial}{\ 2.10}}}\textcolor[HTML]{000000}{\fontsize{12}{12}\selectfont{\global\setmainfont{Arial}{}}}\textcolor[HTML]{000000}{\fontsize{12}{12}\selectfont{\global\setmainfont{Arial}{\textsuperscript{}}}}} & \multicolumn{1}{>{\raggedleft}m{\dimexpr 1.14in+0\tabcolsep}}{\textcolor[HTML]{000000}{\fontsize{12}{12}\selectfont{\global\setmainfont{Arial}{0.03575}}}\textcolor[HTML]{000000}{\fontsize{12}{12}\selectfont{\global\setmainfont{Arial}{}}}\textcolor[HTML]{000000}{\fontsize{12}{12}\selectfont{\global\setmainfont{Arial}{\textsuperscript{}}}}} \\





\multicolumn{1}{>{\raggedright}m{\dimexpr 0.95in+0\tabcolsep}}{\textcolor[HTML]{000000}{\fontsize{12}{12}\selectfont{\global\setmainfont{Arial}{}}}\textcolor[HTML]{000000}{\fontsize{12}{12}\selectfont{\global\setmainfont{Arial}{}}}\textcolor[HTML]{000000}{\fontsize{12}{12}\selectfont{\global\setmainfont{Arial}{$\beta_{2}$}}}\textcolor[HTML]{000000}{\fontsize{12}{12}\selectfont{\global\setmainfont{Arial}{}}}} & \multicolumn{1}{>{\raggedleft}m{\dimexpr 1.4in+0\tabcolsep}}{\textcolor[HTML]{000000}{\fontsize{12}{12}\selectfont{\global\setmainfont{Arial}{-12.321}}}\textcolor[HTML]{000000}{\fontsize{12}{12}\selectfont{\global\setmainfont{Arial}{}}}\textcolor[HTML]{000000}{\fontsize{12}{12}\selectfont{\global\setmainfont{Arial}{\textsuperscript{}}}}} & \multicolumn{1}{>{\raggedleft}m{\dimexpr 1.04in+0\tabcolsep}}{\textcolor[HTML]{000000}{\fontsize{12}{12}\selectfont{\global\setmainfont{Arial}{3.877}}}\textcolor[HTML]{000000}{\fontsize{12}{12}\selectfont{\global\setmainfont{Arial}{}}}\textcolor[HTML]{000000}{\fontsize{12}{12}\selectfont{\global\setmainfont{Arial}{\textsuperscript{}}}}} & \multicolumn{1}{>{\raggedleft}m{\dimexpr 1.13in+0\tabcolsep}}{\textcolor[HTML]{000000}{\fontsize{12}{12}\selectfont{\global\setmainfont{Arial}{-3.18}}}\textcolor[HTML]{000000}{\fontsize{12}{12}\selectfont{\global\setmainfont{Arial}{}}}\textcolor[HTML]{000000}{\fontsize{12}{12}\selectfont{\global\setmainfont{Arial}{\textsuperscript{}}}}} & \multicolumn{1}{>{\raggedleft}m{\dimexpr 1.14in+0\tabcolsep}}{\textcolor[HTML]{000000}{\fontsize{12}{12}\selectfont{\global\setmainfont{Arial}{0.00148}}}\textcolor[HTML]{000000}{\fontsize{12}{12}\selectfont{\global\setmainfont{Arial}{}}}\textcolor[HTML]{000000}{\fontsize{12}{12}\selectfont{\global\setmainfont{Arial}{\textsuperscript{}}}}} \\

\ascline{1.5pt}{666666}{1-5}



\multicolumn{5}{>{\raggedleft}m{\dimexpr 5.65in+8\tabcolsep}}{\textcolor[HTML]{000000}{\fontsize{12}{12}\selectfont{\global\setmainfont{Arial}{0\ <=\ '***'\ <\ 0.001\ <\ '**'\ <\ 0.01\ <\ '*'\ <\ 0.05}}}} \\





\multicolumn{5}{>{\raggedright}m{\dimexpr 5.65in+8\tabcolsep}}{\textcolor[HTML]{000000}{\fontsize{12}{12}\selectfont{\global\setmainfont{Arial}{(Paramètre\ de\ dispersion\ pour\ une\ Binomial\ family:\ 1)}}}\textcolor[HTML]{000000}{\fontsize{12}{12}\selectfont{\global\setmainfont{Arial}{\linebreak }}}\textcolor[HTML]{000000}{\fontsize{12}{12}\selectfont{\global\setmainfont{Arial}{Déviance\ totale\ :\ 126.9\ pour\ 101\ degrés\ de\ liberté}}}\textcolor[HTML]{000000}{\fontsize{12}{12}\selectfont{\global\setmainfont{Arial}{\linebreak }}}\textcolor[HTML]{000000}{\fontsize{12}{12}\selectfont{\global\setmainfont{Arial}{Déviance\ résiduelle\ :\ 109.8\ pour\ 99\ degrés\ de\ liberté}}}\textcolor[HTML]{000000}{\fontsize{12}{12}\selectfont{\global\setmainfont{Arial}{\linebreak }}}\textcolor[HTML]{000000}{\fontsize{12}{12}\selectfont{\global\setmainfont{Arial}{AIC:\ 115.8\ -\ Nombre\ d’itérations\ de\ la\ fonction\ de\ score\ de\ Fisher\ :\ 4}}}} \\





\end{longtable*}



\arrayrulecolor[HTML]{000000}

\global\setlength{\arrayrulewidth}{\Oldarrayrulewidth}

\global\setlength{\tabcolsep}{\Oldtabcolsep}

\renewcommand*{\arraystretch}{1}

Les résultats montrent que le rapport d'aspect et la position verticale
de l'oeil sont tous deux significativement associés au comportement
grégaire (p \textless{} 0,05). Le rapport d'aspect présente un effet
positif, indiquant que les poissons de forme plus élancée ont une
probabilité accrue d'adopter un comportement de banc. À l'inverse, la
position verticale de l'œil montre un effet négatif marqué, suggérant
que les espèces dont l'œil est situé plus haut ont davantage tendance à
un comportement non grégaire.

\begin{Shaded}
\begin{Highlighting}[]
\FunctionTok{chart}\NormalTok{(}\AttributeTok{data =}\NormalTok{ Fish\_data, Comportement\_banc\_bin }\SpecialCharTok{\textasciitilde{}}\NormalTok{ Rapport\_aspect }\SpecialCharTok{\%col=\%}\NormalTok{ Position\_oeil\_vertical ) }\SpecialCharTok{+}
  \FunctionTok{geom\_point}\NormalTok{(}\AttributeTok{size =} \DecValTok{3}\NormalTok{, }\AttributeTok{alpha =} \FloatTok{0.7}\NormalTok{) }\SpecialCharTok{+}
  \FunctionTok{scale\_color\_viridis\_c}\NormalTok{() }\SpecialCharTok{+}
  \FunctionTok{geom\_smooth}\NormalTok{(}
    \AttributeTok{method =} \StringTok{"glm"}\NormalTok{,}
    \AttributeTok{method.args =} \FunctionTok{list}\NormalTok{(}\AttributeTok{family =}\NormalTok{ binomial),}
    \AttributeTok{formula =}\NormalTok{ y }\SpecialCharTok{\textasciitilde{}}\NormalTok{ x,}
    \AttributeTok{se =} \ConstantTok{TRUE}\NormalTok{,}
    \AttributeTok{color =} \StringTok{"green"}\NormalTok{,}
    \AttributeTok{linewidth =} \FloatTok{1.2}
\NormalTok{  ) }\SpecialCharTok{+}
  \FunctionTok{labs}\NormalTok{(}
    \AttributeTok{x =} \StringTok{"Rapport d\textquotesingle{}aspect"}\NormalTok{,}
    \AttributeTok{y =} \StringTok{"Comportement de banc (0 = non, 1 = oui)"}\NormalTok{,}
    \AttributeTok{color =} \StringTok{"Position oeil vertical"}\NormalTok{,}
    \AttributeTok{title =} \StringTok{"Modèle logistique : effet du rapport d\textquotesingle{}aspect et de la position de l\textquotesingle{}oeil"}
\NormalTok{  )}
\end{Highlighting}
\end{Shaded}

\includegraphics{models_report_files/figure-pdf/graphique modele 2-1.pdf}

Le graphique de diagnostic confirme l'absence d'écarts majeurs entre les
valeurs observées et prédites. Ce modèle met ainsi en évidence un lien
fonctionnel entre la morphologie et l'écologie comportementale, reliant
des caractéristiques de forme à des stratégies sociales chez les
poissons.

\section{Discussion et Conclusion}\label{discussion-et-conclusion}

Les résultats obtenus confirment que certains traits morphologiques
jouent un rôle important dans les stratégies de déplacement et de
comportement des poissons. La relation non linéaire observée entre la
longueur maximale et la taille relative de l'œil suit une logique
allométrique bien documentée : les petites espèces investissent
proportionnellement plus dans la vision, ce qui peut être lié à la
prédation ou à la recherche de nourriture. Le modèle polynomial utilisé
explique une part substantielle de la variation, même si une certaine
dispersion résiduelle montre que d'autres facteurs écologiques
interviennent probablement.

Concernant le comportement de banc, le modèle logistique met clairement
en évidence l'influence combinée du rapport d'aspect et de la position
verticale de l'œil. Les poissons au corps plus élancé ont davantage
tendance à se regrouper, ce qui peut s'expliquer par une meilleure
efficacité hydrodynamique en formation serrée. À l'inverse, une position
oculaire plus haute semble associée à un mode de vie moins grégaire,
peut-être lié à une vigilance accrue ou à une écologie plus solitaire.

Dans l'ensemble, ces modèles montrent que la morphologie n'est pas
seulement un reflet de l'évolution, mais aussi une clé pour comprendre
les comportements observés dans les milieux aquatiques. Même si les
analyses restent limitées par les variables disponibles et les
différences d'effectifs entre ordres, elles mettent en lumière des
tendances cohérentes avec la littérature. En conclusion, la morphologie
apparaît comme un bon prédicteur de certaines stratégies de nage, et ces
résultats peuvent contribuer à mieux anticiper la capacité des espèces à
se déplacer ou à franchir des structures dans les environnements
aménagés.

\section{Bibliographie}\label{bibliographie}

Benoit, D., Zielinski, D. P., Swanson, R. A., McLaughlin, R. L.,
Castro-Santos, T., Goodwin, A. E., Pratt, T. C., \& Muir, A. M. (2023).
\emph{FishPass Sortable Attribute Database} (Version 1) {[}Data set{]}.
Dryad. \url{https://doi.org/10.5061/dryad.fqz612jwj}



\end{document}
